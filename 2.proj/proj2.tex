\documentclass[11pt,a4paper,twocolumn]{article}

\usepackage[left=14mm,top=23mm,textwidth=183mm,textheight=252mm]{geometry}
\usepackage{lmodern}
\usepackage[utf8]{inputenc}
\usepackage[T1]{fontenc}
\usepackage[czech]{babel}
\usepackage{amsthm}
\usepackage{enumitem}
\usepackage{textcomp}
\usepackage{amsmath}
\usepackage{amssymb}

\newtheorem{definition}{Definice}
\renewcommand\labelitemi{\large$\bullet$}

\begin{document}

%\maketitle
\begin{titlepage}
    \begin{center}
        {\Huge\textsc{
        Vysoké učení technické v Brně \\
        {\huge Fakulta informačních technologií} \\
        }}
        \vspace{\stretch{0.382}}
        {\LARGE
        Typografie a publikování -- 2. projekt \\ [0.6em]
        Sazba dokumentů a matematických výrazů
        }
        \vspace{\stretch{0.618}}
    \end{center}

    {\Large 2024 \hfill Babušík Michael}
\end{titlepage}

\newpage

\section*{Úvod}
V této úloze si vysázíme titulní stranu a kousek matematického textu, v němž se vyskytují například Definice 1 nebo rovnice (2) na straně 1. Pro vytvoření
těchto odkazů používáme kombinace příkazů \texttt{\textbackslash label},
\texttt{\textbackslash ref}, \texttt{\textbackslash eqref} a \texttt{\textbackslash pageref}. Před odkazy patří nezlomitelná mezera. Pro zvýrazňování textu se používají
příkazy \texttt{\textbackslash verb} a \texttt{\textbackslash emph}.
\par
Titulní strana je vysázena prostředím \texttt{titlepage}
a nadpis je v optickém středu s využitím {\emph přesného} zlatého řezu, který byl probrán na přednášce. Dále jsou na titulní straně čtyři různé velikosti písma a mezi
dvojicemi řádků textu je použito řádkování se zadanou relativní velikostí 0,5\,em a 0,6\,em\footnote{Použijte správný typ mezery mezi číslem a jednotkou.}.

\section{Matematický text}
Matematické symboly a výrazy v plynulém textu jsou
v prostředí \texttt{math}. Definice a věty sázíme v prostředí
definovaném příkazem \texttt{\textbackslash newtheorem} z balíku \texttt{amsthm}.
Tato prostředí obracejí význam \texttt{\textbackslash emph}: uvnitř textu
sázeného kurzívou se zvýrazňuje písmem v základním řezu. Někdy je vhodné použít konstrukci \texttt{\$\{\}\$}
nebo \texttt{\textbackslash mbox\{\}}, která zabrání zalomení (matematického) textu. Pozor také na tvar i sklon řeckých písmen:
srovnejte \texttt{\textbackslash epsilon} a \texttt{\textbackslash varepsilon}, \texttt{\textbackslash Xi} a \texttt{\textbackslash varXi}.

\begin{definition}
    {\upshape Konečný přepisovaci stroj} neboli {\upshape Mealyho automat} je definován jako upořádaná pětice tvaru ${M = (Q, \mathit{\varSigma}, \mathit{\varGamma}, \delta, q_0)}$, kde:
    \begin{itemize}
        \item $Q$ je konečná množina \textup{stavů},
        \item $\mathit{\varSigma}$ je konečná \textup{vstupní abeceda},
        \item $\mathit{\varGamma}$ je konečná \textup{výstupní abeceda},
        \item $\delta:Q \times \mathit{\varSigma} \rightarrow Q \times \mathit{\varGamma}$ je totální \textup{přechodová funkce},
        \item $q_0\in Q$ je \textup{počáteční stav}.
    \end{itemize}
\end{definition}
\subsection{Podsekce s definicí}
Pomocí přechodové funkce $\delta$ zavedeme novou funkci~$\delta^*$ pro překlad vstupních slov $u \in \varSigma^*$ do výstupních slov $w \in \varGamma^*$.

\begin{definition}
    Nechť $M = (Q, \mathit{\varSigma}, \mathit{\varGamma}, \delta, q_0)$ je Mealyho automat. \textup{Překládací funkce} $\delta^*:Q \times \mathit{\varSigma}^* \times \mathit{\varGamma}^* \rightarrow \mathit{\varGamma}^*$ je pro každý stav $q \in Q$, symbol $x \in \mathit{\varSigma}$, slova $u \in \mathit{\varSigma}^*$, $w \in \mathit{\varGamma}^*$ definována rekurentním předpisem:
    \begin{itemize}
        \item $\delta^*(q,\varepsilon, w) = w$
        \item $\delta^*(q, xu, w) = \delta^*(q', u, w y)$, kde $(q', y) = \delta(q,x)$
    \end{itemize}
\end{definition}

\subsection{Rovnice}
Složitější matematické formule sázíme mimo plynulý text pomocí prostředí \texttt{displaymath}. Lze umístit i~vícevýrazů na jeden řádek, ale pak je třeba tyto vhodně oddělit, například pomocí \texttt{\textbackslash quad}, při dostatku místa i~\texttt{\textbackslash qquad}
\begin{displaymath}
    \qquad g^{a_{n}} \notin A^{B^{n}}
    \qquad y^1_0- \sqrt[5]{x+\sqrt[7]{y}}
    \qquad x>y^2>=y^3
\end{displaymath}

Velikost závorek a svislých čar je potřeba přizpůsobit jejich obsahu. Velikost lze stanovit explicitně, a nebo pomocí \texttt{\textbackslash left} a \texttt{\textbackslash right}.
Kombinační čísla sázejte makrem \texttt{\textbackslash binom}.
\begin{displaymath}
    \left| \bigcup P \right| = \sum_{\emptyset \neq X\subseteq P}(-1)^{\left| X \right| -1} \left| \bigcap X \right|
\end{displaymath}

\begin{displaymath}
    F_{n+1} = \binom{n}{0} + \binom{n-1}{1} + \binom{n-2}{2} + \dots + \binom{\lceil \frac{n}{2} \rceil}{\lfloor \frac{n}{2} \rfloor}
\end{displaymath}

V rovnici (1) jsou tři typy závorek s různou \emph{explicitně} definovanou velikostí. Obě rovnice mají svisle zarovnaná rovnítka. Použijte k tomu vhodné prostředí.
\begin{align}
    \biggl( \Bigl\{ b \otimes [c_1 \oplus c_2] \circ a \Bigr\}^\frac{2}{3} \biggr) & \: = \: \log_{z}x                         \\
    \int_{a}^{b} f(x) \, \mathrm{d}x                                               & \: = \: -\int_{b}^{a} f(y) \, \mathrm{d}y
\end{align}
V této větě vidíme, jak se vysází proměnná určující limitu v běžném textu: $\lim_{{m \to \infty}} f(m)$. Podobně je to i s dalšími symboly jako $\bigcup_{N \in \mathcal{M}} N$ či $\sum_{i=1}^{m} x_i^2$. S~vynucením méně úsporné sazby příkazem \texttt{\textbackslash limits} budou vzorce vysázeny v podobě $\lim\limits_{m \to \infty} f(m)$ a $\sum\limits_{i=1}^{m}x_i^2$.

\section{Matice}
Pro sázení matic se používá prostředí \texttt{array} a závorky s výškou nastavenou pomocí \texttt{\textbackslash left}, \texttt{\textbackslash right}.
\[
    D = \left| \begin{array}{cccc}
        a_{11} & a_{12} & \dots  & a_{1n} \\
        a_{21} & a_{22} & \dots  & a_{2n} \\
        \vdots & \vdots & \ddots & \vdots \\
        a_{m1} & a_{m2} & \dots  & a_{mn} \\
    \end{array} \right|
    = \left| \begin{array}{cc}
        x & y \\
        t & w \\
    \end{array} \right|
    = xw - yt
\]

Prostředí \texttt{array} lze úspěšně využít i jinde, například na pravé straně následující rovnosti.
\[
    \binom{n}{k} =
    \left\{
    \begin{array}{ll}
        \frac{n!}{k!(n-k)!} & \text{pro } 0 \leq k \leq n \\
        0                   & \text{jinak}
    \end{array}
    \right.
\]

\end{document}
